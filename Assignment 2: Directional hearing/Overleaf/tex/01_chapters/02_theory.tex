\chapter{Theory}

\section{Head-Related Impulse Response}
\label{sec:hrir}

The Head-Related Impulse Response (HRIR) models how sound is perceived from different directions. The HRIR varies for each person due to differences in head and ear shape, but for this report, we will use a simplified model of the head as a rigid sphere. This model allows us to calculate the time delay between the two ears based on the angle of incidence of the sound wave.

This is called the interaural time difference (ITD). The Woodworth formula is used for calculating this delay at a given angle \( \theta \) and is given by:
\begin{equation}
  \Delta\tau = \frac{a}{c} (\theta + \sin(\theta))
  \label{eq:woodworth}
\end{equation}
where \( a \) is the radius of the head, \( c \) is the speed of sound, and \( \theta \) is the angle of incidence of the sound wave relative to the forward direction of the head \cite{svensson2024compendium}. This returns the time delay \( \Delta\tau \) in seconds, where the sound reaches the ear closer to the source first at \( t = 0 \) and the other ear at \( t = \Delta\tau \).

\section{Head-Related Transfer Function}
\label{sec:hrtf}

The formula for the Head-Related Transfer Function (HRTF) for a rigid sphere model is given by:
\begin{equation}
  H_{L/R}(\omega) = \frac{a_{L/R} \cdot j\omega + \beta}{j\omega + \beta} 
  \label{eq:hrtf}
\end{equation}
where \( a_{L} = 1 - \sin(\theta) \) for the left ear and \( a_{R} = 1 + \sin(\theta) \) for the right ear, \( \beta = \frac{2c}{a} \), \( c \) is the speed of sound, \( a \) is the radius of the head, and \( \theta \) is the angle of incidence of the sound wave relative to the forward direction of the head. This simplified rigid-sphere formulation is based on the model proposed by Brown and Duda (1998) and presented in the Directional Hearing assignment for TTT4295 Acoustic Signal Processing \cite{Svensson2025}.

The HRTF describes how the sound wave is filtered by the head and ears before reaching the ear canal. This can be used to simulate how a sound would be perceived by a listener from different directions.


\section{Bilinear Transform}

To implement the analog HRTF model digitally, the bilinear transform is used to convert the continuous-time transfer function to a discrete-time transfer function. The analog HRTF can be expressed as a first-order low-pass filter:
\begin{equation}
  H(s) = \frac{\alpha_{L/R} s + \beta}{s + \beta},
  \label{eq:lowpass}
\end{equation}
where \( \alpha_L = 1 - \sin(\theta) \) and \( \alpha_R = 1 + \sin(\theta) \) are scaling factors
for the left and right ears, respectively, and \( \beta = \frac{2c}{a} \).

The bilinear transform \cite{BilinearTransform} is defined as:
\begin{equation}
  s = \frac{2}{T} \cdot \frac{z-1}{z+1},
\end{equation}
where \( T = 1/f_s \) is the sampling period.

Applying this to the first-order analog transfer function in (\autoref{eq:lowpass}) yields a discrete-time filter on the form:
\begin{equation}
  H(z) = \frac{B_0 + B_1 z^{-1}}{1 + A_1 z^{-1}},
\end{equation}
with coefficients
\begin{align}
  B_0 &= \frac{2\alpha + \beta T}{2 + \beta T}, &
  B_1 &= \frac{\beta T - 2\alpha}{2 + \beta T}, &
  A_1 &= \frac{\beta T - 2}{2 + \beta T}.
\end{align}

A full derivation of this expression is included in \autoref{app:bilinear}.