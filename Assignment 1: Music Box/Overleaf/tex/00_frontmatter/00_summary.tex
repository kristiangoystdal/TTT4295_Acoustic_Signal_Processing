\chapter*{Summary}

This report presents the analysis of a recording of a music box  using Fast Fourier Transform (FFT) to identify the fundamental frequencies, their harmonics and the corresponding musical notes. The recording was made with a smartphone in WAV format at 44.1 kHz and 16-bit resolution, and the audio was segmented into individual notes for analysis. Each note was zero-padded to achieve an FFT resolution of approximately 0.04 Hz, ensuring accurate frequency estimation.

The harmonic structure of the tones was examined by classifying peaks into harmonically related (Group A) and non-harmonic (Group B) components. The harmonic peaks showed small deviations from their ideal positions, with increasing deviation for higher harmonics. Musical note identification was carried out by mapping the detected fundamental frequencies to equal temperament notes and calculating deviations in cents.

The analysis showed that the music box is generally well-tuned, with most melody and bass notes falling within $\pm40$ cents of their ideal frequencies. While some outliers were observed, especially at higher frequencies, the overall tuning accuracy of the music box was acceptable and consistent with the mechanical limitations. The results highlight both the usefulness of FFT-based analysis for characterizing acoustic signals and the inherent imperfections of mechanical sound sources such as music boxes.


