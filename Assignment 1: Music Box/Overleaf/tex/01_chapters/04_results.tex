\chapter{Results}\label{ch:Results}

The Results chapter should describe all the results obtained in the experiments, and give comments/discussions to the results (but see also what is written in the Discussions chapter). 
\begin{itemize}
\item All kinds of figures/diagrams (figure captions should be centered below the diagram), tables (table captions should be centered above the table), and appendices should be numbered. 
\item {\em Diagrams should be marked with quantities, units and values for both axes}. Measurement series that are compared, or comparisons with calculated values, should be presented in the same diagram. All variables, constants and abbreviations that are not obvious must be explained. 
\item Some types of results are more suitable to present in tables rather than diagrams, as curves.
Large tables should be placed in the appendices. 
\item Measurement results should be commented, with an analysis and an evaluation of the results and measurement data with the aim of drawing a conclusion. If possible, the sources of error should be brought up and their possible effect on the results considered. If relevant, deviations should be quantified and given in percent. As mentioned above, see also what is written in the Discussions chapter.
\item {\em Do not present the same data twice}. If you think a curve is the best way to represent the data, leave out the table containing the same data and vice versa.
\end{itemize}

If you want to present tables, \url{http://www.tablesgenerator.com/} might be of interest. Tables should be referred to as follows and look something like Table \ref{tab:Table}
 
\begin{table}[h] %h (here) can be replaced by H,t (top), and b (bottom) etc. for different positions. This space in between the square brackets can also be left empty for Latex to choose position.
    \centering
    \caption{The caption should be above tables.}    
    \begin{tabular}{lll}
       & Value & Unit                 \\ \hline
    Length & 1     & m                    \\
    Volume & 2.5   & m\textsuperscript{3}
    \label{tab:Table}
    \end{tabular}
\end{table}

