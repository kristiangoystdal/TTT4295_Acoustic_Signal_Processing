\chapter{Discussions}


As shown in \autoref{tab:deviation_segment_1}, we can see that the deviation increases with higher harmonics, indicating that the music box's tuning is not perfectly harmonic.

The harmonic peaks weaken as the frequency increases, as shown in \autoref{tab:harmonic_amplitudes_segment_1}. This is expected, as higher harmonics typically have lower amplitudes due to the physical properties of instruments and the way sound is produced.

In \autoref{tab:melody_freq_notes}, we can see that the melody notes also have some deviation from the ideal frequencies, with deviations ranging from approximately -42.84 to +17.39 cents. This indicates that while the music box is generally well-tuned, there are some notes that are slightly out of tune. The bass notes in \autoref{tab:bass_freq_notes} also show deviations, with values ranging from -37.13 to +42.46 cents. As 100 cents is equal to one semitone, these deviations are relatively small but still noticeable to a trained ear.

Overall, the analysis shows that the music box is reasonably well-tuned, but there are some imperfections in the tuning of both the melody and bass notes. These deviations could be due to various factors, such as the mechanical nature of the music box or slight variations in the manufacturing of the comb teeth. Further analysis could involve comparing these results with other music boxes or exploring methods to improve the tuning accuracy.

