\chapter{Introduction}
%% Erase the text below and enter your introduction 

The reports should be organized with chapters as in the Table of Contents on the previous page, and in the list below. The reasons are: 
\begin{itemize}
\item The reading of the report is made much easier when a standardized structure is used. 
\item This is a very common structure of a report which can be used for most reports on experiments. 
\item It might seem unnecessarily formal that you should present the work like this, when you have in fact been given the task by the reader of the report (the teachers). The idea is that you should practice presenting your measurements in a way which is easily recognizable by the reader - whatever the origin of the measurement task was. 
\end{itemize}
Please note that the report does not have to be long, and the text in each chapter can be very concise. In the following pages, instructions are given on what each chapter should contain.\\

\subsection*{Report chapters: }

\begin{enumerate}
%\item \textbf{Front page} should be as simple as possible. The fonts should be normal and the page should preferably be free of any artistic features. 
\item \textbf{Summary}
\item \textbf{Introduction} 
\item \textbf{Theory} 
\item \textbf{Measurement method and equipment} 
\item \textbf{Results} 
\item \textbf{Discussions} 
\item \textbf{Conclusions} 
\item \textbf{References}
\item \textbf{Appendix} 
\end{enumerate}

\subsection*{In general: }
\begin{itemize}
\item The introducion should describe what is going to be done. In more general research reports, the introduction will also give a motivation for why the work has been done: what are the unclear questions that have not been solved and published elsewhere? What approach is going to be taken in this report? For a lab assignment report, it is not relevant to argue that you have worked un unanswered research questions - but you should explain what is going to be done.

\item Write as a passive 3. person. Focus on what was done, not who did what. 

\item The report should be objective and all statements should be well documented. 

\item Only full sentences should be used. 

\item Be critical about what you include in the report. The reader\textquoteright{}s patience should not be tested. 

\item Never assume that the reader knows about what is written in the report. The report should be able to stand on its own without the reader having to read the assignment text.  

\item You may write in Norwegian. Make sure to change the template to Norwegian headlines and include Norwegian letters if you chose to do so.

\item If in trouble with Latex - google it! There is a massive amount of information out there.

\item In general research reports, the summary and conclusions are often the first parts that are read of a report, and often the only parts of a report that are read. It is therefore important that these parts are clear and consistent. All unnecessary words and vague formulations give a bad impression. 

%\item The lab report should not be too long. More than 10 full pages (excluding front page, table of contents, reference list and appendix) should not be required. Some of the labs might have individual restrictions, make sure to read the exercise text.

\end{itemize}