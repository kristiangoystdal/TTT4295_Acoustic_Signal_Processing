\chapter{Method and Equipment}

\section{Recording setup}

To achieve the recording used for this assignment, a smartphone was used to record the sound of a music box in a quiet room. The phone was placed on a table next to the music box, approximately 10 cm away from it. The recording was done using a application that allows for lossless audio recordings in the WAV format. 

When playing the music box, we made sure that each note from the music box was played separately, allowing for a clear split between the notes in the recording later. Some notes were played at the same time as an accord, which we couldnt avoid due to the setup of the music box. 

\section{Post-processing}


This chapter should describe how the measurements were performed and which equipment was used. The description should be so complete that the measurements could be reconstructed by someone else. The equipment list should contain equipment type, model (often, instrument registration numbers are expected but we don't expect that for these reports). 

A sketch of the setup is often a good way to complement your description of the measurement procedure. A sketch can be seen in Fig.  and should be referred to like this: ... as illustrated in Fig.. A figure should {\em not} be referred to like this: "... as shown in the figure below....."\footnote{One reason for not refering to "the figure below" is that Latex software typically moves figures around and then referring to "the figure below" will not work.}


\section{\label{sec:Measurement-procedure}Measurement procedure}
Explain the measurement procedure. Again, the procedure should be explained in enough detail that someone else could carry out an equivalent experiment.

\section{Equipment}
List the relevant equipment: manufacturer and model number. Often, instrument registration numbers are expected but we don't expect that for these assignment reports.

