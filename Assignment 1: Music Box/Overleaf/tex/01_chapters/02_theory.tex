\chapter{Theory}


\section{FFT} 

Explain the theory behind FFT and how it is used to analyze signals in the frequency domain. Include relevant equations and concepts.

\section{Windowing}

Explain the concept of windowing in signal processing. Discuss different types of windows (e.g., Hamming, Hanning, Blackman) and their effects on spectral leakage and resolution.

\section{Room Impulse Response}

Explain the concept of room impulse response (RIR) and its significance in acoustics. Discuss how RIR can be measured and analyzed.


% The theory section should not be a copy of the assigment text. Use your own words and use literature references as a reference to where more information on a topic can be found. 
% The theory part should only contain the theory which is necessary to interpret and understand the results. The exercise text could be put away and the lab report serves as background material.  You are not expected to present derivations of equations that can be found in other literature. However, if you make any derivations yourself, for instance in your analysis of the results, they should be included.

% All equations should be numbered:
% \begin{equation}
% x^{2}+x-3=0
% \label{eq:Equation}
% \end{equation}
% When writing, you should refer to the equations using \textit{eqref} like this: ... as can be seen in eq.\eqref{eq:Equation}. A rule of thumb is that every equation should be referred to in the text, otherwise it is superfluous. 

